\chapter{Conclusion and Future Work}

In this thesis, we generated a deep neural network potential (NNP) and assessed its predictability in obtaining relevant system properties such as density, surface tension, and dipole orientation. Due to numerical instabilities encountered with the SCAN functional for interfaces, we instead used the similar, but more accurate and numerically efficient, regularized-restored r²SCAN functional. The neural network potential trained solely on bulk systems did not perform well when applied to trajectories containing interfaces, significantly underestimating surface tension and slightly underestimating bulk density. However, adding an interface dataset to the training process improved the prediction of mass density and surface tension, suggesting that surface defects play a role in the surface properties of water. Additionally, the bulk+interface trained NNP achieved the same accuracy as the reference NNP but with a smaller training dataset. Despite this, all models failed to accurately describe microscopic properties such as dipole orientation at the surface, although they were able to obtain good macroscopic properties.

This study should be extended by training the deep neural network model on larger system sizes to account for finite size effects. Furthermore, it would be interesting to explore how including initial parameters, such as the virial, in the training process might affect the results of this thesis. Long-range electrostatic interactions, which play a crucial role in accurately capturing molecular dipolar fluctuations, should also be included during training. Unfortunately, the current DeepMD architecture has limited features for accommodating long-range contributions. Ultimately, the accuracy of the deep neural network potential is closely tied to the DFT accuracy used. Therefore, using a better exchange-correlation functional or pseudopotential could improve the representation of many-body systems.
