\chapter{Conclusion and Future Work}

In this thesis, we generated a deep neural network potential and assess its
predictability in obtaining relevant system properties such as density, surface
tension, and dipole orientation. Due to numerical instabilities using SCAN functional for interfaces, we used instead  similar  but accurate and numerically efficient regularized-restored r$^2$SCAN functional. Neural network potential trained on only bulk
systems did not perform well when used on trajectories containing interfaces.
Specifically, it predicted very underestimated values of surface tension and
slightly underestimated bulk density. Adding interface dataset to the training
did improve the prediction of mass density and surface tension, implying that surface defects
play a role in surface properties of water. In addition, the bulk+interface trained NNP have the same accuracy as the reference NNP used but with less training dataset. In all the models, they fail to describe microscopic properties such as dipole orientation at the surface but still obtain good macroscopic properties. This begs the question whether it is due to finite size effects or rather different physics must be include during the training such as long-range electrostatic interactions.
