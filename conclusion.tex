\chapter{Conclusion and Future Work}

In this thesis, we generated a deep neural network potential and assess its
predictability in obtaining relevant system properties such as density, surface
tension, and dipole orientation. Due to numerical instabilities using SCAN functional for interfaces, we used instead  similar  but accurate and numerically efficient regularized-restored r$^2$SCAN functional. Neural network potential trained on only bulk
systems did not perform well when used on trajectories containing interfaces.
Specifically, it significantly underestimated the surface tension and
slightly underestimated the bulk density. Adding interface dataset to the training
did improve the prediction of mass density and surface tension, implying that surface defects
play a role in surface properties of water. In addition, the bulk+interface trained NNP have the same accuracy as the reference NNP used but with smaller training dataset. In all the models, they fail to describe microscopic properties such as dipole orientation at the surface but still obtain good macroscopic properties.

This study should be extended by training the deep neural network model for large system size as to take into account finite size effects. Moreover, it would be interesting to study how initial parameters such as virial when included in the training will affect the results of this thesis. Also, long-range electrostatic interactions play a major role in obtaining accurate molecular dipolar fluctuations and should be included during training. Unfortunately, the current DeepMD architecture has limited features for long-range contributions. Ultimately, the accuracy of the deep neural network potential is always intertwined to the DFT accuracy used. Hence, a better exchange-correlation functional or pseudopotential should improve the physics of many body systems.
