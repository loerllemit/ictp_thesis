\chapter*{Abstract}
\addcontentsline{toc}{chapter}{Abstract}


In this thesis, we tried using recent computational approach to model many-body potential energy surface by means of application of deep learning technique,  called Deep Neural Molecular Dynamics (DeepMD) implemented in DeePMD-kit. We focus on modelling  neat water interfaces by training the datasets of bulk and bulk+interface systems with DFT level of accuracy  and efficiency of classical force fields. The generated deep neural network potential (NNP) was assessed according to its predictability in obtaining relevant system properties such as mass density, surface tension, and dipole orientation. Due to numerical instabilities using SCAN functional for interfaces, we used instead  similar  but accurate and numerically efficient regularized-restored r$^2$SCAN functional in labelling the datasets. Neural network potential trained on only bulk
systems did not perform well when used on trajectories containing interfaces. Adding interface dataset to the training
did improve the prediction  of macroscopic properties, implying that surface defects
play a role in surface properties of water. In addition, the bulk+interface trained NNP have the same accuracy as the reference NNP used but with less training dataset. In all the models, they fail to describe microscopic properties such as dipole orientation at the surface but still obtain good macroscopic properties.  Finite size effects  and  long-range electrostatic interactions might play a role in obtaining  discrepancies at the microscopic level.
