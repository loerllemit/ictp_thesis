\chapter*{Abstract}
\addcontentsline{toc}{chapter}{Abstract}

This thesis explores the challenges and issues in modeling the many-body potential energy surface for liquid water interfaces by developing a deep neural network potential scheme for molecular dynamics simulations (DPMD), as implemented in DeePMD-kit. The focus is on modeling neat water interfaces by training datasets of bulk and bulk+interface systems with Density Functional Theory (DFT) accuracy and the linear scaling efficiency of empirical force fields. Additionally, DPMD can be parallelized due to its local decomposition and near-neighbor dependence of atomic energies. The generated deep neural network potential (NNP) was assessed for its predictability in obtaining relevant system properties such as mass density, surface tension, and dipole orientation. Due to numerical instabilities using the SCAN functional for interfaces, we instead used a similar but more accurate and numerically efficient regularized-restored r$^2$SCAN functional for labeling the datasets. Neural network potentials trained on only bulk systems did not perform well when applied to trajectories containing interfaces. Adding interface data to the training set improved the prediction of macroscopic properties, implying that surface defects play a role in the surface properties of water. Moreover, the bulk+interface-trained NNP had the same accuracy as the reference NNP but with a smaller training dataset. All models failed to describe microscopic properties such as dipole orientation at the surface but still achieved good macroscopic properties such as surface tension for the bulk+interface-trained NNP. Finite size effects and long-range electrostatic interactions might contribute to discrepancies at the microscopic level.