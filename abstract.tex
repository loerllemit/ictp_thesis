\chapter*{Abstract}
\addcontentsline{toc}{chapter}{Abstract}

This thesis explores the challenges and issues in modelling the many-body potential energy surface
for liquid water interfaces by means of developing deep neural network potential  scheme for molecular dynamics simulations (DPMD), as implemented in DeePMD-kit. We focus on modelling  neat water interfaces by training the datasets of bulk and bulk+interface systems with Density Functional Theory (DFT) level of accuracy and linear scale efficiency of empirical force fields. In addition, DPMD can be parallelized due to its local decomposition and  near-neighbor dependence of  atomic energies. The generated deep neural network potential (NNP) was assessed according to its predictability in obtaining relevant system properties such as mass density, surface tension, and dipole orientation. Due to numerical instabilities using SCAN functional for interfaces, we used instead a similar  but accurate and numerically efficient regularized-restored r$^2$SCAN functional in labelling the datasets. Neural network potential trained on only bulk
systems did not perform well when used on trajectories containing interfaces. Adding interface dataset to the training
did improve the prediction  of macroscopic properties, implying that surface defects
play a role in surface properties of water. In addition, the bulk+interface trained NNP have the same accuracy as the reference NNP used but with a smaller  training dataset. All the models  fail to describe microscopic properties such as dipole orientation at the surface but still obtain good macroscopic properties such as surface tension for the bulk+interface trained NNP.  Finite size effects  and  long-range electrostatic interactions might play a role in obtaining  discrepancies at the microscopic level.
